\documentclass[aspectratio=169, 12pt]{beamer}

% --- Theme and Colors ---
\usetheme{metropolis}
\usepackage{appendixnumberbeamer}
\usepackage{booktabs}
\usepackage[scale=2]{ccicons}
\usepackage{pgfplots}
\usepgfplotslibrary{dateplot}
\usepackage{xspace}
\usepackage{amsmath}

\definecolor{primary}{RGB}{44, 62, 80}
\definecolor{secondary}{RGB}{231, 76, 60}
\setbeamercolor{palette primary}{bg=primary, fg=white}
\setbeamercolor{title separator}{fg=secondary}
\setbeamercolor{progress bar}{fg=secondary, bg=primary}

\title{Unlocking Financial Data}
\subtitle{A Mathematical Journey Through Time Series \& Portfolios}
\date{\today}
\author{Leonardo H. Talero-Sarmiento, Ph.D.}
\institute{Data Science for Finance}

\begin{document}

\maketitle

% ==============================================================================
% SECTION: TIME SERIES DECOMPOSITION
% ==============================================================================
\section{Part I: The Anatomy of a Stock Price}

\begin{frame}{The Starting Point: Raw Price Data}
    \begin{columns}
        \begin{column}{0.5\textwidth}
            \textbf{What do we see here?}
            \begin{itemize}
                \item Chaos? Randomness?
                \item This is Amazon (AMZN) from 2020.
                \item \textit{Question:} If you only look at this, can you make a decision?
                \item \textbf{Problem:} No constant mean, no constant variance.
            \end{itemize}
        \end{column}
        \begin{column}{0.5\textwidth}
            \includegraphics[width=\textwidth]{assets/amzn_price_solo.png}
        \end{column}
    \end{columns}
\end{frame}

\begin{frame}{The Mathematical Framework}
    We assume the price $Y_t$ is a sum of hidden forces:
    \vspace{0.4cm}
    \begin{block}{The Classical Model (Additive)}
        \[ Y_t = T_t + S_t + R_t \]
    \end{block}
    \vspace{0.4cm}
    \begin{itemize}
        \item \textbf{Trend ($T_t$)}: Underlying growth (or decline).
        \item \textbf{Seasonality ($S_t$)}: Periodic cycles (Weekly, Monthly).
        \item \textbf{Residual ($R_t$)}: The "Heartbeat" or Stochastic Error.
    \end{itemize}
\end{frame}

\begin{frame}{Step 1: Extracting the Trend ($T_t$)}
    \begin{columns}
        \begin{column}{0.5\textwidth}
            \includegraphics[width=\textwidth]{assets/amzn_price_solo.png}
            \begin{center}\footnotesize Original Price\end{center}
        \end{column}
        \begin{column}{0.5\textwidth}
            \includegraphics[width=\textwidth]{assets/amzn_trend_only.png}
            \begin{center}\footnotesize The Trend\end{center}
        \end{column}
    \end{columns}
    \vspace{0.3cm}
    \textbf{Analysis:} By using a moving average window, we smooth out volatile days to see the long-term direction. 
\end{frame}

\begin{frame}{Step 2: Hunting for Cycles ($S_t$)}
    \begin{columns}
        \begin{column}{0.5\textwidth}
            \includegraphics[width=\textwidth]{assets/amzn_seasonal_only.png}
        \end{column}
        \begin{column}{0.5\textwidth}
            \textbf{Is it predictable?}
            \begin{itemize}
                \item This pattern repeats every year.
                \item Captures psychological market cycles or reporting periods.
                \item If we subtract this and the Trend, what remains?
            \end{itemize}
        \end{column}
    \end{columns}
\end{frame}

\begin{frame}{Step 3: What is Left? ($R_t$)}
    \begin{columns}
        \begin{column}{0.5\textwidth}
            \includegraphics[width=\textwidth]{assets/amzn_resid_only.png}
        \end{column}
        \begin{column}{0.5\textwidth}
            \textbf{The Stochastic Soul}
            \begin{itemize}
                \item This is the "Noise."
                \item Ideally, it should be **Stationary** (Mean $\approx 0$).
                \item If there is a pattern here, our model is missing something.
                \item \textit{Question:} Can we predict noise? (Hint: No,แต่ we can measure its risk).
            \end{itemize}
        \end{column}
    \end{columns}
\end{frame}

\begin{frame}{The Complete Anatomical View}
    \begin{center}
        \includegraphics[width=0.65\textwidth]{assets/amzn_decomp_clean.png}
    \end{center}
\end{frame}

% ==============================================================================
% SECTION: MODELING & FORECASTING
% ==============================================================================
\section{Part II: Predicting the Unpredictable}

\begin{frame}{How do we project forward?}
    We use the \textbf{ARIMA} model. It assumes the current price depends on its own past.
    \vspace{0.3cm}
    \begin{block}{The ARIMA(p, d, q) Logic}
        \[ y'_t = c + \phi_1 y'_{t-1} + \dots + \theta_1 \epsilon_{t-1} + \dots + \epsilon_t \]
    \end{block}
    \begin{itemize}
        \item \textbf{p (AR)}: How many past days influence today?
        \item \textbf{d (I)}: How many times do we "subtract" to find stationarity?
        \item \textbf{q (MA)}: How much do past "surprises" (errors) matter?
    \end{itemize}
\end{frame}

\begin{frame}{Visualizing the Forecast}
    \begin{columns}
        \begin{column}{0.6\textwidth}
            \includegraphics[width=\textwidth]{assets/amzn_arima.png}
        \end{column}
        \begin{column}{0.4\textwidth}
            \textbf{The Reality Check}
            \begin{itemize}
                \item \textbf{Red Line}: The expectation.
                \item \textbf{Shaded Area}: The uncertainty (95\% Confidence).
                \item Notice how the cone widens as we go further!
            \end{itemize}
        \end{column}
    \end{columns}
\end{frame}

% ==============================================================================
% SECTION: PORTFOLIO THEORY
% ==============================================================================
\section{Part III: Modern Portfolio Theory (MPT)}

\begin{frame}{From One Stock to a Universe}
    \textbf{Why look at Correlation?}
    \begin{columns}
        \begin{column}{0.5\textwidth}
            \includegraphics[width=\textwidth]{assets/portfolio_corr.png}
        \end{column}
        \begin{column}{0.5\textwidth}
            \begin{itemize}
                \item If assets move together (Corr = 1.0), diversification is useless.
                \item If assets move differently (Corr < 0.5), we can "kill" risk.
                \item \textit{Question:} Does AMZN move with MSFT? Usually, yes.
            \end{itemize}
        \end{column}
    \end{columns}
\end{frame}

\begin{frame}{The Optimization Problem}
    We want to find weights $w$ for each stock to minimize risk:
    \vspace{0.3cm}
    \begin{columns}
        \begin{column}{0.5\textwidth}
            \textbf{1. Portfolio Variance:}
            \[ \sigma_p^2 = \sum \sum w_i w_j \sigma_{ij} = w^T \Sigma w \]
            \textbf{2. Portfolio Return:}
            \[ E[R_p] = \sum w_i E[R_i] \]
        \end{column}
        \begin{column}{0.5\textwidth}
            \textbf{The Objective:}
            Maximize the **Sharpe Ratio**:
            \[ S = \frac{E[R_p] - R_{risk\_free}}{\sigma_p} \]
        \end{column}
    \end{columns}
\end{frame}

\begin{frame}{The Efficient Frontier}
    \begin{columns}
        \begin{column}{0.6\textwidth}
            \includegraphics[width=\textwidth]{assets/portfolio_full_frontier.png}
        \end{column}
        \begin{column}{0.4\textwidth}
            \textbf{Choosing your fighter:}
            \begin{itemize}
                \item \textbf{Black Cross}: Minimum Volatility (The "Safety First" choice).
                \item \textbf{Red Star}: Maximum Sharpe (The "Smart Optimizer").
                \item Any point below the curve is inefficient.
            \end{itemize}
        \end{column}
    \end{columns}
\end{frame}

% ==============================================================================
% CLOSING
% ==============================================================================
\begin{frame}{Summary \& Call to Action}
    \begin{enumerate}
        \item \textbf{Anatomize}: Don't look at price, look at its components ($T, S, R$).
        \item \textbf{Optimize}: Don't pick stocks, pick portfolios ($w^T \Sigma w$).
        \item \textbf{Quantify}: Uncertainty is a cone, not a line.
    \end{enumerate}
    \vspace{0.4cm}
    \textbf{Question for you:} 
    Thinking about AMZN's decomposition, do you think the 2024 price is driven more by **Trend** or by **Noise**?
    \vspace{0.4cm}
    \begin{center}
        \huge Thank You!
    \end{center}
\end{frame}

\end{document}
