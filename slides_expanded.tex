\documentclass[aspectratio=169, 12pt]{beamer}

% --- Theme and Colors ---
\usetheme{metropolis}
\usepackage{appendixnumberbeamer}
\usepackage{booktabs}
\usepackage[scale=2]{ccicons}
\usepackage{pgfplots}
\usepgfplotslibrary{dateplot}
\usepackage{xspace}
\usepackage{amsmath}

\definecolor{primary}{RGB}{44, 62, 80}
\definecolor{secondary}{RGB}{231, 76, 60}
\setbeamercolor{palette primary}{bg=primary, fg=white}
\setbeamercolor{title separator}{fg=secondary}
\setbeamercolor{progress bar}{fg=secondary, bg=primary}

\title{Unlocking Financial Data}
\subtitle{A Mathematical Journey Through Time Series \& Portfolios}
\date{\today}
\author{Leonardo H. Talero-Sarmiento, Ph.D.}
\institute{Data Science for Finance}

\begin{document}

\maketitle

% ==============================================================================
% SECTION: TIME SERIES DECOMPOSITION
% ==============================================================================
\section{Part I: The Anatomy of a Stock Price}

\begin{frame}{The Starting Point: Raw Price Data}
    \begin{columns}
        \begin{column}{0.5\textwidth}
            \textbf{What do we see here?}
            \begin{itemize}
                \item Chaos? Randomness?
                \item This is Amazon (AMZN) from 2020.
                \item \textit{Question:} If you only look at this, can you make a decision?
                \item \textbf{Problem:} No constant mean, no constant variance.
            \end{itemize}
        \end{column}
        \begin{column}{0.5\textwidth}
            \includegraphics[width=\textwidth]{assets/amzn_price_solo.png}
        \end{column}
    \end{columns}
\end{frame}

\begin{frame}{The Mathematical Framework}
    We assume the price $Y_t$ is a sum of hidden forces:
    \vspace{0.4cm}
    \begin{block}{The Classical Model (Additive)}
        \[ Y_t = T_t + S_t + R_t \]
    \end{block}
    \vspace{0.4cm}
    \begin{itemize}
        \item \textbf{Trend ($T_t$)}: Underlying growth (or decline).
        \item \textbf{Seasonality ($S_t$)}: Periodic cycles (Weekly, Monthly).
        \item \textbf{Residual ($R_t$)}: The "Heartbeat" or Stochastic Error.
    \end{itemize}
\end{frame}

\begin{frame}{Step 1: Extracting the Trend ($T_t$)}
    \begin{columns}
        \begin{column}{0.5\textwidth}
            \includegraphics[width=\textwidth]{assets/amzn_price_solo.png}
            \begin{center}\footnotesize Original Price\end{center}
        \end{column}
        \begin{column}{0.5\textwidth}
            \includegraphics[width=\textwidth]{assets/amzn_trend_only.png}
            \begin{center}\footnotesize The Trend\end{center}
        \end{column}
    \end{columns}
    \vspace{0.3cm}
    \textbf{Analysis:} By using a moving average window, we smooth out volatile days to see the long-term direction. 
\end{frame}

\begin{frame}{Step 2: Hunting for Cycles ($S_t$)}
    \begin{columns}
        \begin{column}{0.5\textwidth}
            \includegraphics[width=\textwidth]{assets/amzn_seasonal_only.png}
        \end{column}
        \begin{column}{0.5\textwidth}
            \textbf{Is it predictable?}
            \begin{itemize}
                \item This pattern repeats every year.
                \item Captures psychological market cycles or reporting periods.
                \item If we subtract this and the Trend, what remains?
            \end{itemize}
        \end{column}
    \end{columns}
\end{frame}

\begin{frame}{Step 3: What is Left? ($R_t$)}
    \begin{columns}
        \begin{column}{0.5\textwidth}
            \includegraphics[width=\textwidth]{assets/amzn_resid_only.png}
        \end{column}
        \begin{column}{0.5\textwidth}
            \textbf{The Stochastic Soul}
            \begin{itemize}
                \item This is the "Noise."
                \item Ideally, it should be **Stationary** (Mean $\approx 0$).
                \item If there is a pattern here, our model is missing something.
                \item \textit{Question:} Can we predict noise? (Hint: No,แต่ we can measure its risk).
            \end{itemize}
        \end{column}
    \end{columns}
\end{frame}

\begin{frame}{The Complete Anatomical View}
    \begin{center}
        \includegraphics[width=0.65\textwidth]{assets/amzn_decomp_clean.png}
    \end{center}
\end{frame}

% ==============================================================================
% SECTION: MODELING & FORECASTING
% ==============================================================================
\section{Part II: Predicting the Unpredictable}

\begin{frame}{How do we project forward?}
    We use the \textbf{ARIMA} model. It assumes the current price depends on its own past.
    \vspace{0.3cm}
    \begin{block}{The ARIMA(p, d, q) Logic}
        \[ y'_t = c + \phi_1 y'_{t-1} + \dots + \theta_1 \epsilon_{t-1} + \dots + \epsilon_t \]
    \end{block}
    \begin{itemize}
        \item \textbf{p (AR)}: How many past days influence today?
        \item \textbf{d (I)}: How many times do we "subtract" to find stationarity?
        \item \textbf{q (MA)}: How much do past "surprises" (errors) matter?
    \end{itemize}
\end{frame}

\begin{frame}{Visualizing the Forecast}
    \begin{columns}
        \begin{column}{0.6\textwidth}
            \includegraphics[width=\textwidth]{assets/amzn_arima.png}
        \end{column}
        \begin{column}{0.4\textwidth}
            \textbf{The Reality Check}
            \begin{itemize}
                \item \textbf{Red Line}: The expectation.
                \item \textbf{Shaded Area}: The uncertainty (95\% Confidence).
                \item Notice how the cone widens as we go further!
            \end{itemize}
        \end{column}
    \end{columns}
\end{frame}

% ==============================================================================
% SECTION: PORTFOLIO THEORY
% ==============================================================================
\section{Part III: Modern Portfolio Theory (MPT)}

\begin{frame}{From One Stock to a Universe}
    \textbf{Why look at Correlation?}
    \begin{columns}
        \begin{column}{0.5\textwidth}
            \includegraphics[width=\textwidth]{assets/portfolio_corr.png}
        \end{column}
        \begin{column}{0.5\textwidth}
            \begin{itemize}
                \item If assets move together (Corr = 1.0), diversification is useless.
                \item If assets move differently (Corr < 0.5), we can "kill" risk.
                \item \textit{Question:} Does AMZN move with MSFT? Usually, yes.
            \end{itemize}
        \end{column}
    \end{columns}
\end{frame}

% --- PORTFOLIO CONCEPT 1: RETURN ---
\begin{frame}{1. Portfolio Return: The Weighted Average}
    \begin{block}{The Daily Concept}
        Think of your portfolio as a \textbf{smoothie}. If it's 60\% Banana (12\% sugar) and 40\% Kale (8\% sugar), your total sugar content is simply the weighted average of the ingredients.
    \end{block}
    \vspace{0.3cm}
    \textbf{The Equation:}
    \[ E[R_p] = \sum_{i=1}^{n} w_i E[R_i] \]
    \begin{itemize}
        \item $w_i$: The "weight" (percentage of money in stock $i$).
        \item $E[R_i]$: The expected return of stock $i$.
    \end{itemize}
\end{frame}

\begin{frame}{Portfolio Return: Numerical Example}
    \textbf{Case Study: The Tech-Duo}
    \vspace{0.4cm}
    \begin{columns}
        \begin{column}{0.5\textwidth}
            \begin{tabular}{lcc}
                \toprule
                Asset & Weight ($w$) & Return ($R$) \\
                \midrule
                Amazon & 60\% (0.6) & 15\% \\
                Microsoft & 40\% (0.4) & 10\% \\
                \bottomrule
            \end{tabular}
        \end{column}
        \begin{column}{0.5\textwidth}
            \textbf{Calculations:}
            \begin{align*}
                R_p &= (0.6 \times 15\%) + (0.4 \times 10\%) \\
                R_p &= 9\% + 4\% \\
                \mathbf{R_p} &= \mathbf{13\%}
            \end{align*}
        \end{column}
    \end{columns}
    \vspace{0.4cm}
    \textit{Insight:} You can "tune" your return by shifting money to the higher-performing asset.
\end{frame}

% --- PORTFOLIO CONCEPT 2: VARIANCE ---
\begin{frame}{2. Portfolio Variance: The Danger of Co-movement}
    \begin{block}{The Daily Concept}
        Risk is not just the sum of variances. It depends on the \textbf{Correlation}. If you own two umbrella companies, they both fail when it's sunny. If you own an umbrella company AND a sunscreen company, you are protected!
    \end{block}
    \vspace{0.3cm}
    \textbf{The Equation (Matrix Form):}
    \[ \sigma_p^2 = w^T \Sigma w \]
    \textbf{The Equation (2-Asset Form):}
    \[ \sigma_p^2 = w_1^2\sigma_1^2 + w_2^2\sigma_2^2 + 2w_1w_2\mathrm{Cov}_{1,2} \]
\end{frame}

\begin{frame}{Portfolio Variance: Why 'Matching' Matters}
    \textbf{Numerical Impact of Correlation ($\rho$)}
    Imagine Asset A and B both have 20\% risk ($\sigma$).
    \vspace{0.4cm}
    \begin{itemize}
        \item \textbf{Perfect Positive ($\rho = +1.0$)}: Risk is 20\%. No benefit to diversifying.
        \item \textbf{No Correlation ($\rho = 0.0$)}: Risk drops to $\approx 14\%$.
        \item \textbf{Perfect Negative ($\rho = -1.0$)}: Risk drops to \textbf{0\%}.
    \end{itemize}
    \vspace{0.3cm}
    \begin{center}
        \footnotesize \textit{Equation for Covariance: $\mathrm{Cov}_{1,2} = \rho_{1,2} \sigma_1 \sigma_2$}
    \end{center}
\end{frame}

% --- PORTFOLIO CONCEPT 3: SHARPE RATIO ---
\begin{frame}{3. The Sharpe Ratio: Reward vs. Pain}
    \begin{block}{The Daily Concept}
        Is a 20\% return good? Not if you had a 50\% chance of losing everything. The Sharpe Ratio measures the **Return per unit of Risk**.
    \end{block}
    \vspace{0.3cm}
    \textbf{The Equation:}
    \[ S = \frac{R_p - R_{risk\_free}}{\sigma_p} \]
    \begin{itemize}
        \item $R_p - R_f$: The "Excess Return" (The prize).
        \item $\sigma_p$: The "Volatility" (The price you pay in stress/risk).
    \end{itemize}
\end{frame}

\begin{frame}{The Sharpe Ratio: Identifying the Winner}
    \textbf{Numerical Comparison: Which Portfolio is better?}
    \vspace{0.4cm}
    \begin{center}
        \begin{tabular}{lccc}
            \toprule
            Portfolio & Return & Volatility & \textbf{Sharpe Ratio} \\
            \midrule
            A (Aggressive) & 18\% & 25\% & $18/25 = 0.72$ \\
            B (Optimized) & 12\% & 10\% & $12/10 = \mathbf{1.20}$ \\
            \bottomrule
        \end{tabular}
    \end{center}
    \vspace{0.4cm}
    \textbf{Analysis:} Portfolio B is "smarter." It's more efficient. It gives you 1.2 units of return for every unit of risk, while Portfolio A only gives 0.72!
\end{frame}

\begin{frame}{The Efficient Frontier}
    \begin{columns}
        \begin{column}{0.6\textwidth}
            \includegraphics[width=\textwidth]{assets/portfolio_full_frontier.png}
        \end{column}
        \begin{column}{0.4\textwidth}
            \textbf{Choosing your fighter:}
            \begin{itemize}
                \item \textbf{Black Cross}: Minimum Volatility (The "Safety First" choice).
                \item \textbf{Red Star}: Maximum Sharpe (The "Smart Optimizer").
                \item Any point below the curve is inefficient.
            \end{itemize}
        \end{column}
    \end{columns}
\end{frame}

% ==============================================================================
% CLOSING
% ==============================================================================
\begin{frame}{Summary \& Call to Action}
    \begin{enumerate}
        \item \textbf{Anatomize}: Don't look at price, look at its components ($T, S, R$).
        \item \textbf{Optimize}: Don't pick stocks, pick portfolios ($w^T \Sigma w$).
        \item \textbf{Quantify}: Uncertainty is a cone, not a line.
    \end{enumerate}
    \vspace{0.4cm}
    \textbf{Question for you:} 
    Thinking about AMZN's decomposition, do you think the 2024 price is driven more by **Trend** or by **Noise**?
    \vspace{0.4cm}
    \begin{center}
        \huge Thank You!
    \end{center}
\end{frame}

\end{document}
