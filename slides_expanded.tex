\documentclass[aspectratio=169, 12pt]{beamer}

% --- Theme and Colors ---
\usetheme{metropolis}
\usepackage{appendixnumberbeamer}
\usepackage{booktabs}
\usepackage[scale=2]{ccicons}
\usepackage{pgfplots}
\usepgfplotslibrary{dateplot}
\usepackage{xspace}
\usepackage{amsmath}
\usepackage{graphicx} % <- needed for \includegraphics
\usepackage{hyperref} % <- for clickable email

\definecolor{primary}{RGB}{44, 62, 80}
\definecolor{secondary}{RGB}{231, 76, 60}
\setbeamercolor{palette primary}{bg=primary, fg=white}
\setbeamercolor{title separator}{fg=secondary}
\setbeamercolor{progress bar}{fg=secondary, bg=primary}

\title{Unlocking Decisions Based on Financial Data}
\subtitle{Time Series \& Portfolios}
\date{\today}
\author{Leonardo H. Talero-Sarmiento, Ph.D.\\\href{mailto:ltalero@unab.edu.co}{ltalero@unab.edu.co}}
\institute{Introduction to Time-series Modeling in Finance \\ 
\href{https://supervised-data-science.streamlit.app/}{Supervised Data Science App}}




\begin{document}

\maketitle

% Full table of contents after title
\begin{frame}{Outline}
  \tableofcontents
\end{frame}

% ==============================================================================
% SECTION: TIME SERIES DECOMPOSITION
% ==============================================================================
\section{Part I: The Anatomy of a Stock Price}

\begin{frame}{The Starting Point: Raw Price Data}
    \begin{columns}
        \begin{column}{0.5\textwidth}
            \textbf{What do we see here?}
            \begin{itemize}
                \item Chaos? Randomness?
                \item This is Amazon (AMZN) from 2020.
                \item \textit{Question:} If you only look at this, can you make a decision?
                \item \textbf{Problem:} No constant mean, no constant variance.
            \end{itemize}
        \end{column}
        \begin{column}{0.5\textwidth}
            \includegraphics[width=\textwidth]{assets/amzn_price_solo.png}
        \end{column}
    \end{columns}
\end{frame}

\begin{frame}{The Mathematical Framework}
    We assume the price $Y_t$ is a sum of hidden forces:
    \vspace{0.4cm}
    \begin{block}{The Classical Model (Additive)}
        \[ Y_t = T_t + S_t + R_t \]
    \end{block}
    \vspace{0.4cm}
    \begin{itemize}
        \item \textbf{Trend ($T_t$)}: Underlying growth (or decline).
        \item \textbf{Seasonality ($S_t$)}: Periodic cycles (Weekly, Monthly).
        \item \textbf{Residual ($R_t$)}: The ``Heartbeat'' or stochastic error.
    \end{itemize}
\end{frame}

\begin{frame}{Step 1: Extracting the Trend ($T_t$)}
    \begin{columns}
        \begin{column}{0.5\textwidth}
            \includegraphics[width=\textwidth]{assets/amzn_price_solo.png}
            \begin{center}\footnotesize Original Price\end{center}
        \end{column}
        \begin{column}{0.5\textwidth}
            \includegraphics[width=\textwidth]{assets/amzn_trend_only.png}
            \begin{center}\footnotesize The Trend\end{center}
        \end{column}
    \end{columns}
    \vspace{0.3cm}
    \textbf{Analysis:} By using a moving average window, we smooth out volatile days to see the long-term direction.
\end{frame}

\begin{frame}{Step 2: Hunting for Cycles ($S_t$)}
    \begin{columns}
        \begin{column}{0.5\textwidth}
            \includegraphics[width=\textwidth]{assets/amzn_seasonal_only.png}
        \end{column}
        \begin{column}{0.5\textwidth}
            \textbf{Is it predictable?}
            \begin{itemize}
                \item This pattern repeats every year.
                \item Captures psychological market cycles or reporting periods.
                \item If we subtract this and the trend, what remains?
            \end{itemize}
        \end{column}
    \end{columns}
\end{frame}

\begin{frame}{Step 3: What is Left? ($R_t$)}
    \begin{columns}
        \begin{column}{0.5\textwidth}
            \includegraphics[width=\textwidth]{assets/amzn_resid_only.png}
        \end{column}
        \begin{column}{0.5\textwidth}
            \textbf{The Stochastic Soul}
            \begin{itemize}
                \item This is the ``noise.''
                \item Ideally, it should be stationary (mean $\approx 0$).
                \item If there is a pattern here, our model is missing something.
                \item \textit{Question:} Can we predict noise? (Hint: No, but we can measure its risk.)
            \end{itemize}
        \end{column}
    \end{columns}
\end{frame}

\begin{frame}{The Complete Anatomical View}
    \begin{center}
        \includegraphics[width=0.65\textwidth]{assets/amzn_decomp_clean.png}
    \end{center}
\end{frame}

% ==============================================================================
% SECTION: MODELING & FORECASTING
% ==============================================================================
\section{Part II: Predicting the Unpredictable}

\begin{frame}{How do we project forward?}
    We use the \textbf{ARIMA} model. It assumes the current price depends on its own past.
    \vspace{0.3cm}
    \begin{block}{The ARIMA(p, d, q) Logic}
        \[ y'_t = c + \phi_1 y'_{t-1} + \dots + \theta_1 \epsilon_{t-1} + \dots + \epsilon_t \]
    \end{block}
    \begin{itemize}
        \item \textbf{p (AR)}: How many past days influence today?
        \item \textbf{d (I)}: How many times do we ``subtract'' to find stationarity?
        \item \textbf{q (MA)}: How much do past surprises (errors) matter?
    \end{itemize}
\end{frame}

\begin{frame}{Visualizing the Forecast}
    \begin{columns}
        \begin{column}{0.6\textwidth}
            \includegraphics[width=\textwidth]{assets/amzn_arima.png}
        \end{column}
        \begin{column}{0.4\textwidth}
            \textbf{The Reality Check}
            \begin{itemize}
                \item \textbf{Red Line}: The expectation.
                \item \textbf{Shaded Area}: The uncertainty (95\% confidence).
                \item Notice how the cone widens as we go further.
            \end{itemize}
        \end{column}
    \end{columns}
\end{frame}

% ==============================================================================
% SECTION: PORTFOLIO THEORY
% ==============================================================================
\section{Part III: Modern Portfolio Theory (MPT)}

\begin{frame}{From One Stock to a Universe}
    \textbf{Why look at correlation?}
    \begin{columns}
        \begin{column}{0.5\textwidth}
            \includegraphics[width=\textwidth]{assets/portfolio_corr.png}
        \end{column}
        \begin{column}{0.5\textwidth}
            \begin{itemize}
                \item If assets move together (Corr = 1.0), diversification is useless.
                \item If assets move differently (Corr < 0.5), diversification reduces risk.
                \item \textit{Question:} Does AMZN move with MSFT? Usually, yes.
            \end{itemize}
        \end{column}
    \end{columns}
\end{frame}

\begin{frame}{1. Portfolio Return: The Weighted Average}
    \begin{block}{The Daily Concept}
        Think of your portfolio as a \textbf{smoothie}. If it's 60\% banana (12\% sugar) and 40\% kale (8\% sugar), total sugar content is the weighted average.
    \end{block}
    \vspace{0.3cm}
    \textbf{The Equation:}
    \[ E[R_p] = \sum_{i=1}^{n} w_i E[R_i] \]
    \begin{itemize}
        \item $w_i$: weight (portfolio share in asset $i$).
        \item $E[R_i]$: expected return of asset $i$.
    \end{itemize}
\end{frame}

\begin{frame}{Portfolio Return: Numerical Example}
    \textbf{Case Study: The Tech-Duo}
    \vspace{0.4cm}
    \begin{columns}
        \begin{column}{0.5\textwidth}
            \begin{tabular}{lcc}
                \toprule
                Asset & Weight ($w$) & Return ($R$) \\
                \midrule
                Amazon & 60\% (0.6) & 15\% \\
                Microsoft & 40\% (0.4) & 10\% \\
                \bottomrule
            \end{tabular}
        \end{column}
        \begin{column}{0.5\textwidth}
            \textbf{Calculations:}
            \begin{align*}
                R_p &= (0.6 \times 15\%) + (0.4 \times 10\%) \\
                R_p &= 9\% + 4\% \\
                \mathbf{R_p} &= \mathbf{13\%}
            \end{align*}
        \end{column}
    \end{columns}
    \vspace{0.4cm}
    \textit{Insight:} Return shifts with weights.
\end{frame}

\begin{frame}{2. Portfolio Variance: The Danger of Co-movement}
    \begin{block}{The Daily Concept}
        Risk is not just the sum of variances; it depends on \textbf{correlation}. If you own two umbrella companies, both suffer when it is sunny. If you own an umbrella company and a sunscreen company, risk is balanced.
    \end{block}
    \vspace{0.3cm}
    \textbf{The Equation (Matrix Form):}
    \[ \sigma_p^2 = w^T \Sigma w \]
    \textbf{The Equation (2-Asset Form):}
    \[ \sigma_p^2 = w_1^2\sigma_1^2 + w_2^2\sigma_2^2 + 2w_1w_2\mathrm{Cov}_{1,2} \]
\end{frame}

\begin{frame}{Portfolio Variance: Why Matching Matters}
    \textbf{Numerical Impact of Correlation ($\rho$)}
    Imagine Asset A and B both have 20\% volatility ($\sigma$).
    \vspace{0.4cm}
    \begin{itemize}
        \item \textbf{Perfect positive ($\rho = +1.0$)}: risk is 20\%.
        \item \textbf{No correlation ($\rho = 0.0$)}: risk drops to $\approx 14\%$.
        \item \textbf{Perfect negative ($\rho = -1.0$)}: risk drops to \textbf{0\%}.
    \end{itemize}
    \vspace{0.3cm}
    \begin{center}
        \footnotesize \textit{Covariance: $\mathrm{Cov}_{1,2} = \rho_{1,2} \sigma_1 \sigma_2$}
    \end{center}
\end{frame}

\begin{frame}{3. The Sharpe Ratio: Reward vs. Pain}
    \begin{block}{The Daily Concept}
        Is a 20\% return good? Not if the downside risk is extreme. The Sharpe ratio measures return per unit of risk.
    \end{block}
    \vspace{0.3cm}
    \textbf{The Equation:}
    \[ S = \frac{R_p - R_f}{\sigma_p} \]
    \begin{itemize}
        \item $R_p - R_f$: excess return.
        \item $\sigma_p$: volatility.
    \end{itemize}
\end{frame}

\begin{frame}{The Sharpe Ratio: Identifying the Winner}
    \textbf{Numerical Comparison: Which Portfolio is better?}
    \vspace{0.4cm}
    \begin{center}
        \begin{tabular}{lccc}
            \toprule
            Portfolio & Return & Volatility & \textbf{Sharpe Ratio} \\
            \midrule
            A (Aggressive) & 18\% & 25\% & $18/25 = 0.72$ \\
            B (Optimized) & 12\% & 10\% & $12/10 = \mathbf{1.20}$ \\
            \bottomrule
        \end{tabular}
    \end{center}
    \vspace{0.4cm}
    \textbf{Analysis:} Portfolio B is more efficient.
\end{frame}

\begin{frame}{The Efficient Frontier}
    \begin{columns}
        \begin{column}{0.6\textwidth}
            \includegraphics[width=\textwidth]{assets/portfolio_full_frontier.png}
        \end{column}
        \begin{column}{0.4\textwidth}
            \textbf{Choosing your fighter:}
            \begin{itemize}
                \item \textbf{Black Cross}: minimum volatility.
                \item \textbf{Red Star}: maximum Sharpe.
                \item Points below the curve are inefficient.
            \end{itemize}
        \end{column}
    \end{columns}
\end{frame}

% ==============================================================================
% CLOSING
% ==============================================================================
\begin{frame}{Key Takeaways}
    \begin{enumerate}
        \item \textbf{Anatomize}: Do not look only at price; inspect components ($T, S, R$).
        \item \textbf{Optimize}: Do not pick stocks; pick portfolios ($w^T \Sigma w$).
        \item \textbf{Quantify}: Uncertainty is a cone, not a line.
    \end{enumerate}
    \vspace{0.4cm}
    \textbf{Question:}
    Thinking about AMZN's decomposition, is the 2024 price driven more by trend or by noise?
    \vspace{0.4cm}
    \begin{center}
        \huge Thank You!
    \end{center}
\end{frame}


% ==============================================================================
% SECTION: RESEARCH ROADMAP
% ==============================================================================
\section{Part IV: Research Roadmap}

\begin{frame}{Current Articles Under Review and Revision}
    
    \textbf{1. Contemporary Social Science} 
    \textit{“Brewing Degrowth: Regional Adaptations to COVID-19 in the Craft Beer Sector”} 
    Status: Major Revision
    
    \vspace{0.4cm}
    
    \textbf{2. Sex Roles} \textit{“Moral Judgment through the Lens of Gender and Empathy: A Survey Experiment on Moral Foundations”} 
    Status: Under Review
    
    \vspace{0.4cm}
    
    \textbf{3. Expert Systems With Applications} 
    \textit{“A Rolling Two-Stage Stochastic Programming for Cocoa Irrigation Scheduling under Rainfall Uncertainty”} 
    Status: Rejected – New Version in Preparation
    
    \vspace{0.4cm}
    
    \textbf{4. Behaviour \& Information Technology} 
    \textit{“Comparative Study between Heuristic Evaluation Done by Human Experts vs NLP Agents”} 
    Status: Under Review
    
\end{frame}

% ------------------------------------------------------------------------------
% NEW PROJECTS – AGRICULTURE & OPTIMIZATION
% ------------------------------------------------------------------------------

\begin{frame}{Ongoing Projects: Agricultural Systems and Optimization}

\textbf{1. Computers and Electronics in Agriculture (in preparation)} \\
\textit{Tactical Harvest Planning in Olive Oil under Weather Uncertainty: 
a Linear Stochastic Optimization Model with Scenarios Generated by Diffusion Models} \\
Status: Experimental findings completed; manuscript in preparation.

\vspace{0.4cm}

\textbf{2. Agricultural Water Allocation (resubmission)} \\
\textit{Modeling What Matters: A Dual-Phase Framework for Agricultural Water Allocation Based on a Systematic Review of Stochastic Programming Approaches} \\
Status: Rejected; revised version in preparation.

\end{frame}

% ------------------------------------------------------------------------------
% SOCIAL AND LABOR ECONOMICS
% ------------------------------------------------------------------------------

\begin{frame}{Ongoing Projects: Informality and Social Sustainability}

\textbf{1. Employee Responsibilities and Rights Journal (second draft)} \\
\textit{The Precarious Equilibrium: Unpaid Care, Territorial Gaps, and Economic Sustainability in Informal Footwear SMEs} \\
Status: Second draft in progress.

\vspace{0.4cm}

\textbf{2. Agricultural Sustainability} \\
\textit{Agricultural Sustainability Performance in the EU-27} \\
Status: Experimental phase completed; drafting manuscript; target journal under identification (Agricultural Economics).

\end{frame}

% ------------------------------------------------------------------------------
% ENGINEERING & PREDICTIVE SYSTEMS
% ------------------------------------------------------------------------------

\begin{frame}{Ongoing Projects: Engineering and Predictive Modeling}

\textbf{1. Measurement and Control (preliminary findings)} \\
\textit{Hybrid Evolutionary Optimization for Turbofan Remaining Useful Life Prediction and Fractional-Order Control Tuning} \\
Status: Preliminary findings completed. \\
\vspace{0.4cm}
\textbf{2. Journal of Management Decision } \\ 
\textit{A multi-objective fuzzy Optimization model for Resource Management in an Emergency Department} \\
Status: Finishing experimental phase and crafting first draft. 

\end{frame}

% ------------------------------------------------------------------------------
% MORALITY, MARKETING, AND BEHAVIOR
% ------------------------------------------------------------------------------

\begin{frame}{Ongoing Projects: Moral Cognition and Markets}

\textbf{1. Marketing and Moral Foundations (beer advertising)} \\
\textit{Moral Foundations Theory in Super Bowl Beer Marketing Ideation} \\
Status: Working manuscript.

\vspace{0.4cm}

\textbf{2. Moral Psychology (submission in preparation)} \\
\textit{Beyond Harm: How Distinct Empathy Components Shape Moral Severity} \\
Status: Submission in preparation.

\end{frame}

% ------------------------------------------------------------------------------
% REGIONAL VIOLENCE AND SOCIAL ANALYSIS
% ------------------------------------------------------------------------------

\begin{frame}{Ongoing Projects: Violence and Social Perception}

\textbf{1. Orquídeas Project – Norte de Santander} \\
\textit{Violence Perception and Social Dynamics in Norte de Santander: Evidence from University Student Surveys} \\
Status: Manuscript in development.

\end{frame}



\end{document}